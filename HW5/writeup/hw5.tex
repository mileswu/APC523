\documentclass[11pt]{article}
\setlength{\topmargin}{0in}
\setlength{\headheight}{0in}
\setlength{\headsep}{0in}
\setlength{\textheight}{8.7in}
\setlength{\textwidth}{6.5in}
\setlength{\oddsidemargin}{0in}
\setlength{\evensidemargin}{0in}
\setlength{\parindent}{0.15in}
\setlength{\parskip}{0.05in}

\usepackage{graphicx}
\usepackage{subfig}

\begin{document}

\title{APC523: HW5}
\author{Miles Wu}
\maketitle

\section{Question 1}
For the scalar advection equation $u_t + a u_x = 0$, we know the analytical equation, which is simply that the function moves right with speed $v$: $u(x, t) = u(x -  a t, 0)$. This makes it ideal to test our numerical methods as we can compare it to the exact solution. In our case $a=1$, and we evolve to $t=20$. This corresponds to running for exactly 5 periods, so the analytic solution at $t=20$ is identical to the initial function.

We run Lax-Friedrichs, Lax-Wendroff and Upwind method on various CFL, where the CFL parameter is defined by $a \delta t / \delta x$. Figure 1--3 show this with a CFL of 1.0, 0.8 and 0.5 respectively.

When CFL$=1$, we find that all there methods reproduce the analytical solution essentially perfectly. This s because the grid point spacing divided by the timestep is exactly equal to the velocity of the wave, so that the methods do not need to do any interpolation of the wavefront lying between two grid points.

When CFL$=0.8$, we no longer find that the evolved solution is close to the original function. The upwind and Lax-Freidrichs method have caused the top hat function to diffuse, so it has this rounded shape as the wall regions near $x=\pm 1$ have diffused outwards, but it keeps the central peak correctly around the origin. For the Lax-Wendroff method, it stays a little bit closer to the shape of the top hat function because it is a second-order method, but it also might be a little off in position (hard to tell). The sinusoidal bumps in the solution is because the Lax-Wendroff method is dispersive; the top hat when Fourier decomposed is made up of many sine waves of different frequencies and each frequency wave propagates at a different speed, causing the solution to spread out and break apart a little. This is what we see with the sinusoidal bumps as these are wave components moving apart.

When CFL$=0.5$ it is a similar situation to CFL$=0.8$, only more extreme.

\begin{figure}
  \centering
\input{q1-cfl1.0.tex}
  \caption{Lax-Friedrichs, Lax-Wendroff and Upwind method solutions for the scalar advection equation applied to a top hat function of width 2 evolved to $t=20$. Here it uses a CFL$=1.0$}
  \label{multigrid-plot}
\end{figure}

\begin{figure}
  \centering
\input{q1-cfl0.8.tex}
  \caption{Lax-Friedrichs, Lax-Wendroff and Upwind method solutions for the scalar advection equation applied to a top hat function of width 2 evolved to $t=20$. Here it uses a CFL$=0.8$}
  \label{multigrid-plot}
\end{figure}

\begin{figure}
  \centering
\input{q1-cfl0.5.tex}
  \caption{Lax-Friedrichs, Lax-Wendroff and Upwind method solutions for the scalar advection equation applied to a top hat function of width 2 evolved to $t=20$. Here it uses a CFL$=0.5$}
  \label{multigrid-plot}
\end{figure}


\section{Question 2}

\begin{figure}
  \centering
\input{q2.tex}
  \caption{The L1 error in the Lax-Friedrichs, Lax-Wendroff and Upwind method solutions for the scalar advection equation applied to a sine wave evolved for one period using CFL$=0.5$ for varying grid sizes.}
  \label{multigrid-plot}
\end{figure}
We calculated the L1 error as:
\begin{eqnarray}
	\textnormal{L1 Error} = \frac{1}{N} \sum_{i=0}^N |u_i - u_i^0|
\end{eqnarray}
Figure 4 has the L1 error plotted against the number of grid points from 16 to 1024 in powers of two on a log-log scale. There was a small complication I encountered when making this figure at first, where the lines did not end up straight and were very zig-zag-y. This it turns out was because the timestep ($\delta t = \delta x \times\textnormal{CFL} / a$) was not a factor of $1.0$, so the evolution did not end exactly in the right place corresponding to a single period. The solution I found was to work out the time remainder ($1.0 - t_f$) and to generate a CFL corresponding to this timestep (CFL$= a \delta t / \delta x$) and to use this CFL in the iteration for the final step. This got us to exactly $t=1$.

It is clear that Lax-Wendroff is most accurate by a significant margin. Upwind is slightly better than Lax-Friedrichs too.

The convergence rates are just the negative of the slope of the lines on the log-log plot. These are listed in the table below. It is clear from this that Lax-Wendroff is significantly better as it appears to have quadratic convergence (second-order), but both Lax-Freidrichs and Upwind have the same convergence rate and it seems to be linear convergence (first-order).

\begin{center}
\begin{tabular}{|l |c|}
\hline
Method & Convergence Rate \\
\hline
	Lax-Friedrichs & 0.89 \\
	Upwind & 0.92 \\
	Lax-Wendroff & 1.97 \\
\hline
\end{tabular}
\end{center}


\end{document}