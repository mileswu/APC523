\documentclass[11pt]{article}
\setlength{\topmargin}{0in}
\setlength{\headheight}{0in}
\setlength{\headsep}{0in}
\setlength{\textheight}{8.7in}
\setlength{\textwidth}{6.5in}
\setlength{\oddsidemargin}{0in}
\setlength{\evensidemargin}{0in}
\setlength{\parindent}{0.15in}
\setlength{\parskip}{0.05in}

\usepackage{graphicx}
\usepackage{subfig}

\begin{document}

\title{APC523: HW4}
\author{Miles Wu}
\maketitle

\section{Problem}
The equation we need to solve is:
\begin{eqnarray}
u''(x) &=& -20 + 60 \pi x \cos (20 \pi x^3) - 1800 \pi^2 x^4 \sin (20 \pi x^3)
\label{origeqn}
\end{eqnarray}
with boundary conditions of $u(0) = 1$ and $u(1) = 3$. If we use a finite centered difference approximation on a grid with 255 points we can describe the double derivative of $u$ by the following:
\begin{eqnarray}
u''_i = \frac{1}{h^2} ( u_{i+1} - 2 u_i + u_{i-1} )
\end{eqnarray}

Now that we have 255 simultaneous equations (one for each grid point), we can use Jacobi iteration to solve this system of linear equations. Our initial guess will be $u^0(x) = 1 + 2x$.

\section{Singlegrid}
\begin{figure}
  \centering
\input{singlegrid.tex}
  \caption{Numerical solutions to Equation~\ref{origeqn} using Jacobi iteration and the finite difference method on a single grid are plotted in addition to the analytical solution. Three different number of iterations are plotted.}
  \label{singlegrid-plot}
\end{figure}
First we just use a single grid of 255 points to solve Equation~\ref{origeqn}, using 20, 100 and 1000 iterations of the Jacobi method. The results are shown in Figure~\ref{singlegrid-plot}. Whilst it captures much of the short wavelength behaviour (e.g. from 0.5 to 1.0), it is very far off the analytical solution because the long wavelength behaviour has not converged yet at these number of iterations.

In order to get close to the analytical solution around 100000 iterations are needed which is somewhat computationally expensive (takes approximately half a minute). Most problems are in three dimensions which would increase the complexity dramatically making this unfeasible. This is the motivation for looking for a better algorithm.

\section{Multigrid}
\begin{figure}
  \centering
\input{multigrid.tex}
  \caption{A numerical solution to Equation~\ref{origeqn} using Jacobi iteration and the finite difference method using multiple grids is plotted in addition to the analytical solution. Seven levels of grids were used, with 20 iterations on each level.}
  \label{multigrid-plot}
\end{figure}
Next we use a multigrid method. Here we first of all solve on the 255 grid for ten iterations and calculate the residual. The residual is then coarsened to a smaller grid and ten more iterations are performed on the residual. This is recursed for seven grid levels. Once we reach the 3 grid, the corrections are then interpolated back up to the finer grid doing ten more iterations to smooth out any interpolation roughness. This is recursed back up to the 255 grid. This is known as a V-cycle.

Figure~\ref{multigrid-plot} shows the results of our multigrid method, where we used 7 grids and 20 iterations on each level (10 on the way down and 10 on the way up). First it is immediately obvious that this is much more accurate than single grid method, even though it is using only 140 iterations ($7\times 20$) which is much less than the 1000 iterations of the single grid. This is because the long wavelength behaviour has converged.

Here we see the essence of multigrid. On the small grids the long wavelengths become short wavelengths and hence converge quickly. On the large grids the short wavelengths converge quickly. We just combine the results from the different levels, making sure to iterate one more set to smooth out interpolation errors.

Also worth noting is that it runs exceedingly quickly (around 10 milliseconds). Because many of the Jacobi iterations are run on smaller grids, it has to do much less than 140 full iterations. By comparing runtimes, 140 iterations on the multigrid is about the same runtime as 30 iterations on the singlegrid. This makes the multigrid substantially more efficient than the single grid. 

\end{document}